\documentclass[11pt,a4paper]{article}

% Packages
\usepackage[utf8]{inputenc}
\usepackage[spanish, es-tabla]{babel}
\usepackage{caption}
\usepackage{listings}
\usepackage{adjustbox}
\usepackage{enumitem}
\usepackage{boldline}
\usepackage{amssymb, amsmath}
\usepackage{amsthm}
\usepackage[margin=1in]{geometry}
\usepackage{xcolor}
\usepackage{soul}
\usepackage{hyperref}

% Meta
\title{Metodologías Agiles}
\author{Grupo 2.2 }
\date{}

% Custom
\providecommand{\abs}[1]{\lvert#1\rvert}
\setlength\parindent{0pt}
\definecolor{Light}{gray}{.90}
\newcommand\ddfrac[2]{\frac{\displaystyle #1}{\displaystyle #2}}
% Primera derivada parcial: \pder[f]{x}
\newcommand{\pder}[2][]{\frac{\partial#1}{\partial#2}}

\begin{document}
\maketitle

Como parte de la elaboración de las prácticas, hemos estudiado, comprendido y sentido en carne propia las metodologías ágiles. De ellas sentimos que hemos recibido cosas buenas, y que estas han sido en general positivas. En este documento expondremos beneficios y perjuicios de estas.

\section{Beneficios}

En general hemos encontrado que la planificación ágil es mucho más agradable de realizar. Además de ser más humana promueve un modo de promover las tareas. En particular:

\begin{itemize}
    \item Nos ha gustado la elicitación de roles. Son tareas que deben realizarse y que está bien que se sepa desde un inicio quien tiene la autoridad y el deber de realizarlas.

	\item Durante la fase de planificación inicial el estar centrados en historias de usuario y el cliente antes de los requisitos que nos imaginemos nos pareció muy buena idea. Las técnicas de empatía como los casos de uso y escenarios ayudan a que nos preocupamos de lo que verdad importa: las cosas que añaden utilidad y valor.
	
    \item Una vez sabiamos que queriamos hacer, la planificación en historias de usuario y tareas. Nos gustó mucho Kanban, es una manera muy entretenida de ver el progreso y motivar al equipo. Coincidimos con XP en que el espacio de trabajo debe ser informativo. 
    
    \item Las técnicas de planificación y reparto de trabajo nos fueron especialmente útiles a la hora de realizar el trabajo. Especialmente relevantes a la hora de realizar un reparto equitativo de la carga de trabajo.
    
    \item De la metodología scrum nos gustó otra vez los roles para dar responsabilidad. Además las standup son muy provechosas.
    
    \item Las técnicas de creatividad nos fueron muy útiles a la hora de generar y seleccionar ideas. Fueron nuestro principal fuente de inspiración.
    
    \item Nos fue muy útil el hincapié que hacen las metodologías útiles en la comunicación del equipo. Facilitó las prácticas e hizo mucho más fluido el desarrollo.
    
    \item Afrontar cambios fue mucho más sencillo mediante las técnicas que propone scrum. Es algo que hemos podido experimentar durante las prácticas.
    
    \item Por último los valores que promueven estás metodologías son parecen muy agradables y aunque muchos parezcan básicos, son necesarios para un desarrollo mejor del trabajo.
\end{itemize}

\section{Perjuicios}

\begin{itemize}

	\item El método de trabajo que propone scrum es radicalmente opuesto al que hemos llevado hasta ahora. Es por ello que la adaptación ha sido difícil al principio.
	
	\item Muchas veces la mala aplicación de las técnicas de scrum, como predicción de tiempo de trabajo, nos condujo a pérdidas de eficiencia. Este tipo de errores son principalmente debidos al desconocimiento de las técnicas y la poca experiencia. 

\end{itemize}

\end{document}