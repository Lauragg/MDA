\documentclass[11pt,a4paper]{article}

% Packages
\usepackage[utf8]{inputenc}
\usepackage[spanish, es-tabla]{babel}
\usepackage{caption}
\usepackage{listings}
\usepackage{adjustbox}
\usepackage{enumitem}
\usepackage{boldline}
\usepackage{amssymb, amsmath}
\usepackage{amsthm}
\usepackage[margin=1in]{geometry}
\usepackage{xcolor}
\usepackage{soul}
\usepackage{hyperref}

% Meta
\title{Metodologías Agiles}
\author{Grupo 2.2 }
\date{}

% Custom
\providecommand{\abs}[1]{\lvert#1\rvert}
\setlength\parindent{0pt}
\definecolor{Light}{gray}{.90}
\newcommand\ddfrac[2]{\frac{\displaystyle #1}{\displaystyle #2}}
% Primera derivada parcial: \pder[f]{x}
\newcommand{\pder}[2][]{\frac{\partial#1}{\partial#2}}

\begin{document}
\maketitle

Como parte de la elaboración de las prácticas, hemos estudiado, comprendido y sentido en carne propia las metodologías ágiles. De ellas sentimos que hemos recibido cosas buenas, y que estas han sido en general positivas. En este documento expondremos beneficios y perjuicios de estas.

\section{Beneficios}

En general hemos encontrado que la planificación ágil es mucho más agradable de realizar. Además de ser más humana promueve un modo de promover las tareas. En particular:

\begin{itemize}
    \item Nos ha gustado la elicitación de roles. Son tareas que deben realizarse y que está bien que se sepa desde un inicio quien tiene la autoridad y el deber de realizarlas.

	\item Durante la fase de planificación inicial el estar centrados en historias de usuario y el cliente antes de los requisitos que nos imaginemos nos pareció muy buena idea. Las técnicas de empatía como los casos de uso y escenarios ayudan a que nos preocupamos de lo que verdad importa: las cosas que añaden utilidad y valor.
	
    \item Una vez sabiamos que queriamos hacer, la planificación en historias de usuario y tareas. Nos gustó mucho Kanban, es una manera muy entretenida de ver el progreso y motivar al equipo. Coincidimos con XP en que el espacio de trabajo debe ser informativo. 
    
    \item De la metodología scrum nos gustó otra vez los roles para dar responsabilidad. Además las standup son muy provechosas.
    
    \item Por último los valores que promueven estás metodologías son parecen muy agradables y aunque muchos parezcan básicos, son necesarios.
\end{itemize}

\section{Perjuicios}

\end{document}