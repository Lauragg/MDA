%%
% Copyright (c) 2017 - 2019, Pascal Wagler;
% Copyright (c) 2014 - 2019, John MacFarlane
%
% All rights reserved.
%
% Redistribution and use in source and binary forms, with or without
% modification, are permitted provided that the following conditions
% are met:
%
% - Redistributions of source code must retain the above copyright
% notice, this list of conditions and the following disclaimer.
%
% - Redistributions in binary form must reproduce the above copyright
% notice, this list of conditions and the following disclaimer in the
% documentation and/or other materials provided with the distribution.
%
% - Neither the name of John MacFarlane nor the names of other
% contributors may be used to endorse or promote products derived
% from this software without specific prior written permission.
%
% THIS SOFTWARE IS PROVIDED BY THE COPYRIGHT HOLDERS AND CONTRIBUTORS
% "AS IS" AND ANY EXPRESS OR IMPLIED WARRANTIES, INCLUDING, BUT NOT
% LIMITED TO, THE IMPLIED WARRANTIES OF MERCHANTABILITY AND FITNESS
% FOR A PARTICULAR PURPOSE ARE DISCLAIMED. IN NO EVENT SHALL THE
% COPYRIGHT OWNER OR CONTRIBUTORS BE LIABLE FOR ANY DIRECT, INDIRECT,
% INCIDENTAL, SPECIAL, EXEMPLARY, OR CONSEQUENTIAL DAMAGES (INCLUDING,
% BUT NOT LIMITED TO, PROCUREMENT OF SUBSTITUTE GOODS OR SERVICES;
% LOSS OF USE, DATA, OR PROFITS; OR BUSINESS INTERRUPTION) HOWEVER
% CAUSED AND ON ANY THEORY OF LIABILITY, WHETHER IN CONTRACT, STRICT
% LIABILITY, OR TORT (INCLUDING NEGLIGENCE OR OTHERWISE) ARISING IN
% ANY WAY OUT OF THE USE OF THIS SOFTWARE, EVEN IF ADVISED OF THE
% POSSIBILITY OF SUCH DAMAGE.
%%

%%
% This is the Eisvogel pandoc LaTeX template.
%
% For usage information and examples visit the official GitHub page:
% https://github.com/Wandmalfarbe/pandoc-latex-template
%%

% Options for packages loaded elsewhere
\PassOptionsToPackage{unicode}{hyperref}
\PassOptionsToPackage{hyphens}{url}
\PassOptionsToPackage{dvipsnames,svgnames*,x11names*,table}{xcolor}
%
\documentclass[
  a4paper,
,tablecaptionabove
]{scrartcl}
\usepackage{lmodern}
\usepackage{graphicx}
\usepackage{setspace}
\setstretch{1.2}
\usepackage{amssymb,amsmath}
\usepackage{ifxetex,ifluatex}
\ifnum 0\ifxetex 1\fi\ifluatex 1\fi=0 % if pdftex
  \usepackage[T1]{fontenc}
  \usepackage[utf8]{inputenc}
  \usepackage{textcomp} % provide euro and other symbols
\else % if luatex or xetex
  \usepackage{unicode-math}
  \defaultfontfeatures{Scale=MatchLowercase}
  \defaultfontfeatures[\rmfamily]{Ligatures=TeX,Scale=1}
\fi
% Use upquote if available, for straight quotes in verbatim environments
\IfFileExists{upquote.sty}{\usepackage{upquote}}{}
\IfFileExists{microtype.sty}{% use microtype if available
  \usepackage[]{microtype}
  \UseMicrotypeSet[protrusion]{basicmath} % disable protrusion for tt fonts
}{}
\makeatletter
\@ifundefined{KOMAClassName}{% if non-KOMA class
  \IfFileExists{parskip.sty}{%
    \usepackage{parskip}
  }{% else
    \setlength{\parindent}{0pt}
    \setlength{\parskip}{6pt plus 2pt minus 1pt}}
}{% if KOMA class
  \KOMAoptions{parskip=half}}
\makeatother
\usepackage{xcolor}
\definecolor{default-linkcolor}{HTML}{A50000}
\definecolor{default-filecolor}{HTML}{A50000}
\definecolor{default-citecolor}{HTML}{4077C0}
\definecolor{default-urlcolor}{HTML}{4077C0}
\IfFileExists{xurl.sty}{\usepackage{xurl}}{} % add URL line breaks if available
\IfFileExists{bookmark.sty}{\usepackage{bookmark}}{\usepackage{hyperref}}
\hypersetup{
  linkcolor=blue,
  urlcolor=blue,
  hidelinks,
  breaklinks=true,
  pdfcreator={LaTeX via pandoc with the Eisvogel template}}
\urlstyle{same} % disable monospaced font for URLs
    \usepackage[margin=2.5cm,includehead=true,includefoot=true,centering]{geometry}
  \usepackage{listings}
\newcommand{\passthrough}[1]{#1}
\lstset{defaultdialect=[5.3]Lua}
\lstset{defaultdialect=[x86masm]Assembler}
\usepackage{longtable,booktabs}
% Correct order of tables after \paragraph or \subparagraph
\usepackage{etoolbox}
\makeatletter
\patchcmd\longtable{\par}{\if@noskipsec\mbox{}\fi\par}{}{}
\makeatother
% Allow footnotes in longtable head/foot
\IfFileExists{footnotehyper.sty}{\usepackage{footnotehyper}}{\usepackage{footnote}}
\makesavenoteenv{longtable}
\setlength{\emergencystretch}{3em}  % prevent overfull lines
\providecommand{\tightlist}{%
  \setlength{\itemsep}{0pt}\setlength{\parskip}{0pt}}
\setcounter{secnumdepth}{-\maxdimen} % remove section numbering

% Make use of float-package and set default placement for figures to H.
% The option H means 'PUT IT HERE' (as  opposed to the standard h option which means 'You may put it here if you like').
\usepackage{float}
\floatplacement{figure}{H}


\makeatletter
\newcommand*\bigcdot{\mathpalette\bigcdot@{.76}}
\newcommand*\bigcdot@[2]{\mathbin{\vcenter{\hbox{\scalebox{#2}{$\m@th#1\bullet$}}}}}
\makeatother

\date{}



%%
%% added
%%

%
% language specification
%
% If no language is specified, use English as the default main document language.
%

\ifnum 0\ifxetex 1\fi\ifluatex 1\fi=0 % if pdftex
  \usepackage[shorthands=off,main=english]{babel}
\else
    % Workaround for bug in Polyglossia that breaks `\familydefault` when `\setmainlanguage` is used.
  % See https://github.com/Wandmalfarbe/pandoc-latex-template/issues/8
  % See https://github.com/reutenauer/polyglossia/issues/186
  % See https://github.com/reutenauer/polyglossia/issues/127
  \renewcommand*\familydefault{\sfdefault}
    % load polyglossia as late as possible as it *could* call bidi if RTL lang (e.g. Hebrew or Arabic)
  \usepackage{polyglossia}
  \setmainlanguage[]{english}
\fi


%
% for the background color of the title page
%

%
% break urls
%
\PassOptionsToPackage{hyphens}{url}

%
% When using babel or polyglossia with biblatex, loading csquotes is recommended
% to ensure that quoted texts are typeset according to the rules of your main language.
%
\usepackage{csquotes}

%
% captions
%
\definecolor{caption-color}{HTML}{777777}
\usepackage[font={stretch=1.2}, textfont={color=caption-color}, position=top, skip=4mm, labelfont=bf, singlelinecheck=false, justification=raggedright]{caption}
\setcapindent{0em}

%
% blockquote
%
\definecolor{blockquote-border}{RGB}{221,221,221}
\definecolor{blockquote-text}{RGB}{119,119,119}
\usepackage{mdframed}
\newmdenv[rightline=false,bottomline=false,topline=false,linewidth=3pt,linecolor=blockquote-border,skipabove=\parskip]{customblockquote}
\renewenvironment{quote}{\begin{customblockquote}\list{}{\rightmargin=0em\leftmargin=0em}%
\item\relax\color{blockquote-text}\ignorespaces}{\unskip\unskip\endlist\end{customblockquote}}

%
% Source Sans Pro as the de­fault font fam­ily
% Source Code Pro for monospace text
%
% 'default' option sets the default
% font family to Source Sans Pro, not \sfdefault.
%
\usepackage[default]{sourcesanspro}
\usepackage{sourcecodepro}

% XeLaTeX specific adjustments for straight quotes: https://tex.stackexchange.com/a/354887
% This issue is already fixed (see https://github.com/silkeh/latex-sourcecodepro/pull/5) but the
% fix is still unreleased.
% TODO: Remove this workaround when the new version of sourcecodepro is released on CTAN.
\ifxetex
\makeatletter
\defaultfontfeatures[\ttfamily]
  { Numbers   = \sourcecodepro@figurestyle,
    Scale     = \SourceCodePro@scale,
    Extension = .otf }
\setmonofont
  [ UprightFont    = *-\sourcecodepro@regstyle,
    ItalicFont     = *-\sourcecodepro@regstyle It,
    BoldFont       = *-\sourcecodepro@boldstyle,
    BoldItalicFont = *-\sourcecodepro@boldstyle It ]
  {SourceCodePro}
\makeatother
\fi

%
% heading color
%
\definecolor{heading-color}{RGB}{40,40,40}
\addtokomafont{section}{\color{heading-color}}
% When using the classes report, scrreprt, book,
% scrbook or memoir, uncomment the following line.
%\addtokomafont{chapter}{\color{heading-color}}

%
% variables for title and author
%
\usepackage{titling}
\title{}
\author{}

%
% tables
%

\definecolor{table-row-color}{HTML}{F5F5F5}
\definecolor{table-rule-color}{HTML}{999999}

%\arrayrulecolor{black!40}
\arrayrulecolor{table-rule-color}     % color of \toprule, \midrule, \bottomrule
\setlength\heavyrulewidth{0.3ex}      % thickness of \toprule, \bottomrule
\renewcommand{\arraystretch}{1.3}     % spacing (padding)

% Reset rownum counter so that each table
% starts with the same row colors.
% https://tex.stackexchange.com/questions/170637/restarting-rowcolors
\let\oldlongtable\longtable
\let\endoldlongtable\endlongtable
\renewenvironment{longtable}{
\rowcolors{3}{}{table-row-color!100}  % row color
\oldlongtable} {
\endoldlongtable
\global\rownum=0\relax}

% Unfortunately the colored cells extend beyond the edge of the
% table because pandoc uses @-expressions (@{}) like so:
%
% \begin{longtable}[]{@{}ll@{}}
% \end{longtable}
%
% https://en.wikibooks.org/wiki/LaTeX/Tables#.40-expressions

%
% remove paragraph indention
%
\setlength{\parindent}{0pt}
\setlength{\parskip}{6pt plus 2pt minus 1pt}
\setlength{\emergencystretch}{3em}  % prevent overfull lines

%
%
% Listings
%
%


%
% listing colors
%
\definecolor{listing-background}{HTML}{F7F7F7}
\definecolor{listing-rule}{HTML}{B3B2B3}
\definecolor{listing-numbers}{HTML}{B3B2B3}
\definecolor{listing-text-color}{HTML}{000000}
\definecolor{listing-keyword}{HTML}{435489}
\definecolor{listing-identifier}{HTML}{435489}
\definecolor{listing-string}{HTML}{00999A}
\definecolor{listing-comment}{HTML}{8E8E8E}
\definecolor{listing-javadoc-comment}{HTML}{006CA9}

\lstdefinestyle{eisvogel_listing_style}{
  language         = java,
  numbers          = left,
  xleftmargin      = 2.7em,
  framexleftmargin = 2.5em,
  backgroundcolor  = \color{listing-background},
  basicstyle       = \color{listing-text-color}\small\ttfamily{}\linespread{1.15}, % print whole listing small
  breaklines       = true,
  frame            = single,
  framesep         = 0.19em,
  rulecolor        = \color{listing-rule},
  frameround       = ffff,
  tabsize          = 4,
  numberstyle      = \color{listing-numbers},
  aboveskip        = 1.0em,
  belowskip        = 0.1em,
  abovecaptionskip = 0em,
  belowcaptionskip = 1.0em,
  keywordstyle     = \color{listing-keyword}\bfseries,
  classoffset      = 0,
  sensitive        = true,
  identifierstyle  = \color{listing-identifier},
  commentstyle     = \color{listing-comment},
  morecomment      = [s][\color{listing-javadoc-comment}]{/**}{*/},
  stringstyle      = \color{listing-string},
  showstringspaces = false,
  escapeinside     = {/*@}{@*/}, % Allow LaTeX inside these special comments
  literate         =
  {á}{{\'a}}1 {é}{{\'e}}1 {í}{{\'i}}1 {ó}{{\'o}}1 {ú}{{\'u}}1
  {Á}{{\'A}}1 {É}{{\'E}}1 {Í}{{\'I}}1 {Ó}{{\'O}}1 {Ú}{{\'U}}1
  {à}{{\`a}}1 {è}{{\'e}}1 {ì}{{\`i}}1 {ò}{{\`o}}1 {ù}{{\`u}}1
  {À}{{\`A}}1 {È}{{\'E}}1 {Ì}{{\`I}}1 {Ò}{{\`O}}1 {Ù}{{\`U}}1
  {ä}{{\"a}}1 {ë}{{\"e}}1 {ï}{{\"i}}1 {ö}{{\"o}}1 {ü}{{\"u}}1
  {Ä}{{\"A}}1 {Ë}{{\"E}}1 {Ï}{{\"I}}1 {Ö}{{\"O}}1 {Ü}{{\"U}}1
  {â}{{\^a}}1 {ê}{{\^e}}1 {î}{{\^i}}1 {ô}{{\^o}}1 {û}{{\^u}}1
  {Â}{{\^A}}1 {Ê}{{\^E}}1 {Î}{{\^I}}1 {Ô}{{\^O}}1 {Û}{{\^U}}1
  {œ}{{\oe}}1 {Œ}{{\OE}}1 {æ}{{\ae}}1 {Æ}{{\AE}}1 {ß}{{\ss}}1
  {ç}{{\c c}}1 {Ç}{{\c C}}1 {ø}{{\o}}1 {å}{{\r a}}1 {Å}{{\r A}}1
  {€}{{\EUR}}1 {£}{{\pounds}}1 {«}{{\guillemotleft}}1
  {»}{{\guillemotright}}1 {ñ}{{\~n}}1 {Ñ}{{\~N}}1 {¿}{{?`}}1
  {…}{{\ldots}}1 {≥}{{>=}}1 {≤}{{<=}}1 {„}{{\glqq}}1 {“}{{\grqq}}1
  {”}{{''}}1
}
\lstset{style=eisvogel_listing_style}

\lstdefinelanguage{XML}{
  morestring      = [b]",
  moredelim       = [s][\bfseries\color{listing-keyword}]{<}{\ },
  moredelim       = [s][\bfseries\color{listing-keyword}]{</}{>},
  moredelim       = [l][\bfseries\color{listing-keyword}]{/>},
  moredelim       = [l][\bfseries\color{listing-keyword}]{>},
  morecomment     = [s]{<?}{?>},
  morecomment     = [s]{<!--}{-->},
  commentstyle    = \color{listing-comment},
  stringstyle     = \color{listing-string},
  identifierstyle = \color{listing-identifier}
}

%
% header and footer
%
\usepackage{fancyhdr}

\fancypagestyle{eisvogel-header-footer}{
  \fancyhead{}
  \fancyfoot{}
  \lhead[]{}
  \chead[]{}
  \rhead[]{}
  \lfoot[\thepage]{}
  \cfoot[]{}
  \rfoot[]{\thepage}
  \renewcommand{\headrulewidth}{0.4pt}
  \renewcommand{\footrulewidth}{0.4pt}
}
\pagestyle{eisvogel-header-footer}

%%
%% end added
%%


\begin{document}

%%
%% begin titlepage
%%

%%
%% end titlepage
%%



\hypertarget{metodologuxedas-de-desarrollo-uxe1gil}{%
\section{Metodologías de Desarrollo
Ágil}\label{metodologuxedas-de-desarrollo-uxe1gil}}

\hypertarget{plan-de-entregas}{%
\subsection{Plan de entregas}\label{plan-de-entregas}}

\hypertarget{grupo-2.2}{%
\subsubsection{Grupo 2.2}\label{grupo-2.2}}

Enlace a GitHub: 
\hypertarget{Enlace a Github}{%
\url{https://github.com/Lauragg/MDA/projects}\label{Enlace a Github}}

\hypertarget{breve-descripciuxf3n-del-alcance-del-sistema}{%
\subsection{1. Breve descripción del alcance del
sistema}\label{breve-descripciuxf3n-del-alcance-del-sistema}}

Para el proyecto de \enquote{\textbf{Tu ciudad}} planteamos la
implementación de una aplicación para móvil que nos ayude a encontrar
distintos servicios de interés en la ciudad de Granada.

Los objetivos más importantes que debe tener nuestra app son: + Un mapa
en el que podamos localizar los distintos servicios. + Una lista en la
que se vayan mostrando los distintos servicios. + Poder localizar
contenedores y puntos limpios. + Poder localizar fuentes. + Poder
localizar parques.

Para más información sobre la descripción del proyecto, consultar el
\textbf{\emph{Documento de visión del producto}}.

\hypertarget{listado-inicial-de-historias-de-usuario}{%
\subsection{2. Listado inicial de Historias de
Usuario}\label{listado-inicial-de-historias-de-usuario}}

\textbf{Alta prioridad}

\begin{longtable}[]{@{}lll@{}}
\toprule
ID & Título & PH\tabularnewline
\midrule
\endhead
1 & Usuario - ver mapa funcional con los servicios & 5\tabularnewline
2 & Usuario - buscar servicios mediante un buscador & 4\tabularnewline
3 & Usuario - localizar fuentes & 2\tabularnewline
4 & Usuario - localizar parques & 2\tabularnewline
5 & Administrador - crear información de un servicio & 2\tabularnewline
6 & Administrador - modificar información de un servicio &
1\tabularnewline
7 & Administrador - eliminar información de un servicio &
1\tabularnewline
8 & Usuario - ver una lista con los servicios & 2\tabularnewline
9 & Usuario - localizar contenedores & 2\tabularnewline
\bottomrule
\end{longtable}

\textbf{Media prioridad}

\begin{longtable}[]{@{}lll@{}}
\toprule
ID & Título & PH\tabularnewline
\midrule
\endhead
10 & Usuario - distinguir tipos de contenedores & 1,5\tabularnewline
11 & Usuario - distinguir tipos de parques & 1,5\tabularnewline
12 & Usuario - consultar si un servicio está en mal estado &
1,5\tabularnewline
13 & Usuario - consultar información de un servicio & 1,5\tabularnewline
14 & Usuario - marcar si algo no funciona & 1,5\tabularnewline
\bottomrule
\end{longtable}

\textbf{Baja prioridad}

\begin{longtable}[]{@{}lll@{}}
\toprule
ID & Título & PH\tabularnewline
\midrule
\endhead
15 & Usuario - obtener ruta óptima a un servicio & 4\tabularnewline
16 & Usuario - recibir notificaciones sobre servicios marcados &
2\tabularnewline
17 & Usuario - registrarse & 3\tabularnewline
18 & Usuario - vincular la aplicación con servicios externos &
2\tabularnewline
\bottomrule
\end{longtable}

\hypertarget{cuxe1lculo-de-la-velocidad-del-equipo}{%
\subsection{3. Cálculo de la velocidad del
equipo}\label{cuxe1lculo-de-la-velocidad-del-equipo}}

El equipo está formado por 5 programadores que van a dedicar parte de su
tiempo al proyecto.

Cada iteración del proyecto, es de 2 semanas de duración. La estimación
de esfuerzo viene expresada en días reales de programación.

Duración de una iteración, 1 iteración = 2 semanas = 10 días reales de
trabajo.

La velocidad de equipo en puntos de historia, se ha estimado entre 15 y
18 PH por iteración.

\hypertarget{descripciuxf3n-de-las-entregas}{%
\subsection{4. Descripción de las
entregas}\label{descripciuxf3n-de-las-entregas}}

\textbf{Primer sprint}

\begin{longtable}[]{@{}lll@{}}
\toprule
ID & Título & PH\tabularnewline
\midrule
\endhead
1 & Usuario - ver un mapa funcional con los servicios & 5\tabularnewline
8 & Usuario - ver una lista con los servicios & 2\tabularnewline
9 & Usuario - localizar contenedores & 2\tabularnewline
3 & Usuario - localizar fuentes & 2\tabularnewline
4 & Usuario - localizar parques & 2\tabularnewline
\bottomrule
\end{longtable}

\textbf{Segundo sprint}

\begin{longtable}[]{@{}lll@{}}
\toprule
ID & Título & PH\tabularnewline
\midrule
\endhead
2 & Usuario - buscar servicios mediante un buscador & 4\tabularnewline
14 & Usuario - marcar si algo no funciona & 1,5\tabularnewline
13 & Usuario - consultar información de un servicio & 1,5\tabularnewline
12 & Usuario - consultar si un servicio está en mal estado &
1,5\tabularnewline
11 & Usuario - distintguir tipos de parque & 1,5\tabularnewline
10 & Usuario - distintguir tipos de contenedores & 1,5\tabularnewline
7 & Administrador - eliminar información de un servicio &
1\tabularnewline
6 & Administrador - modificar información de un servicio &
1\tabularnewline
5 & Administrador - crear información de un servicio & 2\tabularnewline
\bottomrule
\end{longtable}

\textbf{Tercer sprint}

\begin{longtable}[]{@{}lll@{}}
\toprule
ID & Título & PH\tabularnewline
\midrule
\endhead
1 & Usuario - vincular la aplicación con servicios externos &
2\tabularnewline
8 & Usuario - registrarse & 3\tabularnewline
9 & Usuario - recibir notificaciones sobre servicios marcados &
2\tabularnewline
3 & Usuario - obtener ruta óptima para cada servicio & 4\tabularnewline
\bottomrule
\end{longtable}

\hypertarget{tarjetas-de-las-hu}{%
\subsection{5. Tarjetas de las HU}\label{tarjetas-de-las-hu}}

\begin{tabular*}{\textwidth}{@{\extracolsep{\fill}} |l|l|l|l|l|l|}

  \hline
   Identificador: 1 \ \ \ &\multicolumn{5}{l|}{Ver un mapa con los servicios }\\
   
   \hline
   \multicolumn{6}{|l|}{ \textbf{Descripción:} }\\ 
   \multicolumn{6}{|l|}{  Mostrar un mapa interactivo con iconos que indican los servicios seleccionados, su tipo }\\ 
   \multicolumn{6}{|l|}{y su localización.  }\\ 
   \hline
  Estimación: 5& Prioridad: alta&\multicolumn{4}{c|}{Entrega: 1} \\ \hline
   \multicolumn{6}{|l|}{ \textbf{Pruebas de aceptación:} }\\ 
   
   \multicolumn{6}{|l|}{ \ \ \ \ $\bigcdot$  Mostrar la localización del usuario en base a los datos GPS y sus alrededores.}\\ 
   \multicolumn{6}{|l|}{ \ \ \ \ $\bigcdot$  Deslizar hacia la derecha para abrir el menú lateral.}\\
   \multicolumn{6}{|l|}{ \ \ \ \ $\bigcdot$  Seleccionar un elemento del mapa para mostrar información sobre el mismo.}\\
   \multicolumn{6}{|l|}{ \ \ \ \ $\bigcdot$  Pulsar el icono de lista para ver la lista.}\\ 
  
  
  \hline 
   \multicolumn{6}{|l|}{ \textbf{Observaciones:} }\\ 
   \multicolumn{6}{|l|}{ El mapa es el estado inicial de la aplicación.}\\ \hline
   
  
  
\end{tabular*}

\begin{tabular*}{\textwidth}{@{\extracolsep{\fill}} |l|l|l|l|l|l|}

  \hline
   Identificador: 8 \ \ \ &\multicolumn{5}{l|}{Ver una lista con los servicios }\\
   
   \hline
   \multicolumn{6}{|l|}{ \textbf{Descripción:} }\\ 
   \multicolumn{6}{|l|}{  Mostrar una lista con los servicios disponibles.}\\ 
   %\multicolumn{6}{|l|}{  Aquí se inlcuye unda descripción con 2 lineas.}\\ 
   \hline
  Estimación: 2& Prioridad: alta&\multicolumn{4}{c|}{Entrega: 1} \\ \hline
   \multicolumn{6}{|l|}{ \textbf{Pruebas de aceptación:} }\\ 
   
   \multicolumn{6}{|l|}{ \ \ \ \ $\bigcdot$  Pulsar el icono de mapa para volver al mapa.}\\ 
   \multicolumn{6}{|l|}{ \ \ \ \ $\bigcdot$  Al seleccionar un elemento en la lista, mostrar un icono para verlo en el mapa.}\\
   \multicolumn{6}{|l|}{ \ \ \ \ $\bigcdot$  Al pulsar el icono de vista de mapa de un elemento, se mostrará el elemento en el mapa.}\\
   \multicolumn{6}{|l|}{ \ \ \ \ $\bigcdot$  Al seleccionar un elemento en la lista, mostrar un icono para ver su información.}\\
   \multicolumn{6}{|l|}{ \ \ \ \ $\bigcdot$  Al pulsar el icono de información de un elemento, se mostrará la información del elemento.}\\
   \multicolumn{6}{|l|}{ \ \ \ \ $\bigcdot$  Deslizar hacia la derecha para abrir el menú lateral.}\\
   
  
  
  \hline 
   \multicolumn{6}{|l|}{ \textbf{Observaciones:} }\\ 
   \multicolumn{6}{|l|}{  Por defecto, la lista se muestra en orden alfabético y por categorías.}\\ \hline
   
  
  
\end{tabular*}

\begin{tabular*}{\textwidth}{@{\extracolsep{\fill}} |l|l|l|l|l|l|}

  \hline
   Identificador: 3 \ \ \ &\multicolumn{5}{l|}{Localizar fuentes }\\
   
   \hline
   \multicolumn{6}{|l|}{ \textbf{Descripción:} }\\ 
   \multicolumn{6}{|l|}{  Mostrar fuentes tanto en la lista como en el mapa.}\\ 
   %\multicolumn{6}{|l|}{  Aquí se inlcuye unda descripción con 2 lineas.}\\ 
   \hline
  Estimación: 2& Prioridad: alta&\multicolumn{4}{c|}{Entrega: 1} \\ \hline
   \multicolumn{6}{|l|}{ \textbf{Pruebas de aceptación:} }\\ 
   
   \multicolumn{6}{|l|}{ \ \ \ \ $\bigcdot$  Al seleccionar la categoría \textit{fuentes}, mostrar únicamente las fuentes en el mapa.}\\
   \multicolumn{6}{|l|}{ \ \ \ \ $\bigcdot$  Al seleccionar la categoría \textit{fuentes}, mostrar únicamente las fuentes en la lista.}\\ 
  
  
  \hline 
   \multicolumn{6}{|l|}{ \textbf{Observaciones:} }\\ 
   \multicolumn{6}{|l|}{  }\\ \hline
   
  
  
\end{tabular*}

\begin{tabular*}{\textwidth}{@{\extracolsep{\fill}} |l|l|l|l|l|l|}

  \hline
   Identificador: 9 \ \ \ &\multicolumn{5}{l|}{Localizar contenedores }\\
   
   \hline
   \multicolumn{6}{|l|}{ \textbf{Descripción:} }\\ 
   \multicolumn{6}{|l|}{  Mostrar fuentes tanto en la lista como en el mapa.}\\ 
   %\multicolumn{6}{|l|}{  Aquí se inlcuye unda descripción con 2 lineas.}\\ 
   \hline
  Estimación: 2& Prioridad: alta&\multicolumn{4}{c|}{Entrega: 1} \\ \hline
   \multicolumn{6}{|l|}{ \textbf{Pruebas de aceptación:} }\\ 
   
   \multicolumn{6}{|l|}{ \ \ \ \ $\bigcdot$  Al seleccionar la categoría \textit{fuentes}, mostrar únicamente los contenedores en el mapa.}\\
   \multicolumn{6}{|l|}{ \ \ \ \ $\bigcdot$  Al seleccionar la categoría \textit{fuentes}, mostrar únicamente los contenedores en la lista.}\\ 
  
  
  \hline 
   \multicolumn{6}{|l|}{ \textbf{Observaciones:} }\\ 
   \multicolumn{6}{|l|}{  }\\ \hline
   
  
  
\end{tabular*}

\begin{tabular*}{\textwidth}{@{\extracolsep{\fill}} |l|l|l|l|l|l|}

  \hline
   Identificador: 4 \ \ \ &\multicolumn{5}{l|}{Localizar parques }\\
   
   \hline
   \multicolumn{6}{|l|}{ \textbf{Descripción:} }\\ 
   \multicolumn{6}{|l|}{  Mostrar fuentes tanto en la lista como en el mapa.}\\ 
   %\multicolumn{6}{|l|}{  Aquí se inlcuye unda descripción con 2 lineas.}\\ 
   \hline
  Estimación: 2& Prioridad: alta&\multicolumn{4}{c|}{Entrega: 1} \\ \hline
   \multicolumn{6}{|l|}{ \textbf{Pruebas de aceptación:} }\\ 
   
   \multicolumn{6}{|l|}{ \ \ \ \ $\bigcdot$  Al seleccionar la categoría \textit{fuentes}, mostrar únicamente los parques en el mapa.}\\
   \multicolumn{6}{|l|}{ \ \ \ \ $\bigcdot$  Al seleccionar la categoría \textit{fuentes}, mostrar únicamente los parques en la lista.}\\ 
  
  
  \hline 
   \multicolumn{6}{|l|}{ \textbf{Observaciones:} }\\ 
   \multicolumn{6}{|l|}{  }\\ \hline
   
  
  
\end{tabular*}

\end{document}
