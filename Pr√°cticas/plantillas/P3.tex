\documentclass[]{article}
\usepackage{lmodern}
\usepackage{amssymb,amsmath}
\usepackage{ifxetex,ifluatex}
\usepackage{fixltx2e} % provides \textsubscript
\ifnum 0\ifxetex 1\fi\ifluatex 1\fi=0 % if pdftex
  \usepackage[T1]{fontenc}
  \usepackage[utf8]{inputenc}
\else % if luatex or xelatex
  \ifxetex
    \usepackage{mathspec}
  \else
    \usepackage{fontspec}
  \fi
  \defaultfontfeatures{Ligatures=TeX,Scale=MatchLowercase}
\fi
% use upquote if available, for straight quotes in verbatim environments
\IfFileExists{upquote.sty}{\usepackage{upquote}}{}
% use microtype if available
\IfFileExists{microtype.sty}{%
\usepackage[]{microtype}
\UseMicrotypeSet[protrusion]{basicmath} % disable protrusion for tt fonts
}{}
\PassOptionsToPackage{hyphens}{url} % url is loaded by hyperref
\usepackage[unicode=true]{hyperref}
\hypersetup{
            pdfborder={0 0 0},
            breaklinks=true}
\urlstyle{same}  % don't use monospace font for urls
\usepackage{longtable,booktabs}
% Fix footnotes in tables (requires footnote package)
\IfFileExists{footnote.sty}{\usepackage{footnote}\makesavenoteenv{long table}}{}
\IfFileExists{parskip.sty}{%
\usepackage{parskip}
}{% else
\setlength{\parindent}{0pt}
\setlength{\parskip}{6pt plus 2pt minus 1pt}
}
\setlength{\emergencystretch}{3em}  % prevent overfull lines
\providecommand{\tightlist}{%
  \setlength{\itemsep}{0pt}\setlength{\parskip}{0pt}}
\setcounter{secnumdepth}{0}
% Redefines (sub)paragraphs to behave more like sections
\ifx\paragraph\undefined\else
\let\oldparagraph\paragraph
\renewcommand{\paragraph}[1]{\oldparagraph{#1}\mbox{}}
\fi
\ifx\subparagraph\undefined\else
\let\oldsubparagraph\subparagraph
\renewcommand{\subparagraph}[1]{\oldsubparagraph{#1}\mbox{}}
\fi

% set default figure placement to htbp
\makeatletter
\def\fps@figure{htbp}
\makeatother


\date{}

\begin{document}

\section{Metodologías de Desarrollo
Ágil}\label{metodologuxedas-de-desarrollo-uxe1gil}

\subsection{Plan de entregas}\label{plan-de-entregas}

\subsubsection{Grupo 2.2}\label{grupo-2.2}

\subsection{1. Breve descripción del alcance del
sistema}\label{breve-descripciuxf3n-del-alcance-del-sistema}

Para el proyecto de ``\textbf{Tu ciudad}'' planteamos la implementación
de una aplicación para móvil que nos ayude a encontrar distintos
servicios de interés en la ciudad de Granada.

Los objetivos más importantes que debe tener nuestra app son: + Un mapa
en el que podamos localizar los distintos servicios. + Una lista en la
que se vayan mostrando los distintos servicios. + Poder localizar
contenedores y puntos limpios. + Poder localizar fuentes. + Poder
localizar parques.

Para más información sobre la descripción del proyecto, consultar el
\textbf{\emph{Documento de visión del producto}}.

\subsection{2. Listado inicial de Historias de
Usuario}\label{listado-inicial-de-historias-de-usuario}

\textbf{Alta prioridad}

\begin{longtable}[]{@{}lll@{}}
\toprule
ID & Título & PH\tabularnewline
\midrule
\endhead
1 & Usuario - ver mapa funcional con los servicios & 5\tabularnewline
2 & Usuario - buscar servicios mediante un buscador & 4\tabularnewline
3 & Usuario - localizar fuentes & 2\tabularnewline
4 & Usuario - localizar parques & 2\tabularnewline
5 & Administrador - crear información de un servicio & 2\tabularnewline
6 & Administrador - modificar información de un servicio &
1\tabularnewline
7 & Administrador - eliminar información de un servicio &
1\tabularnewline
8 & Usuario - ver una lista con los servicios & 2\tabularnewline
9 & Usuario - localizar contenedores & 2\tabularnewline
\bottomrule
\end{longtable}

\textbf{Media prioridad}

\begin{longtable}[]{@{}lll@{}}
\toprule
ID & Título & PH\tabularnewline
\midrule
\endhead
10 & Usuario - distinguir tipos de contenedores & 1,5\tabularnewline
11 & Usuario - distinguir tipos de parques & 1,5\tabularnewline
12 & Usuario - consultar si un servicio está en mal estado &
1,5\tabularnewline
13 & Usuario - consultar información de un servicio & 1,5\tabularnewline
14 & Usuario - marcar si algo no funciona & 1,5\tabularnewline
\bottomrule
\end{longtable}

\textbf{Baja prioridad}

\begin{longtable}[]{@{}lll@{}}
\toprule
ID & Título & PH\tabularnewline
\midrule
\endhead
15 & Usuario - obtener ruta óptima a un servicio & 4\tabularnewline
16 & Usuario - recibir notificaciones sobre servicios marcados &
2\tabularnewline
17 & Usuario - registrarse & 3\tabularnewline
18 & Usuario - vincular la aplicación con servicios externos &
2\tabularnewline
\bottomrule
\end{longtable}

\subsection{3. Cálculo de la velocidad del
equipo}\label{cuxe1lculo-de-la-velocidad-del-equipo}

El equipo está formado por 5 programadores que van a dedicar parte de su
tiempo al proyecto.

Cada iteración del proyecto, es de 2 semanas de duración. La estimación
de esfuerzo viene expresada en días reales de programación.

Duración de una iteración, 1 iteración = 2 semanas = 10 días reales de
trabajo.

La velocidad de equipo en puntos de historia, se ha estimado entre 15 y
18 PH por iteración.

\subsection{4. Descripción de las
entregas}\label{descripciuxf3n-de-las-entregas}

\textbf{Primer sprint}

\begin{longtable}[]{@{}lll@{}}
\toprule
ID & Título & PH\tabularnewline
\midrule
\endhead
1 & Usuario - ver un mapa funcional con los servicios & 5\tabularnewline
8 & Usuario - ver una lista con los servicios & 2\tabularnewline
9 & Usuario - localizar contenedores & 2\tabularnewline
3 & Usuario - localizar fuentes & 2\tabularnewline
4 & Usuario - localizar parques & 2\tabularnewline
\bottomrule
\end{longtable}

\textbf{Segundo sprint}

\begin{longtable}[]{@{}lll@{}}
\toprule
ID & Título & PH\tabularnewline
\midrule
\endhead
2 & Usuario - buscar servicios mediante un buscador & 4\tabularnewline
14 & Usuario - marcar si algo no funciona & 1,5\tabularnewline
13 & Usuario - consultar información de un servicio & 1,5\tabularnewline
12 & Usuario - consultar si un servicio está en mal estado &
1,5\tabularnewline
11 & Usuario - distintguir tipos de parque & 1,5\tabularnewline
10 & Usuario - distintguir tipos de contenedores & 1,5\tabularnewline
7 & Administrador - eliminar información de un servicio &
1\tabularnewline
6 & Administrador - modificar información de un servicio &
1\tabularnewline
5 & Administrador - crear información de un servicio & 2\tabularnewline
\bottomrule
\end{longtable}

\textbf{Tercer sprint}

\begin{longtable}[]{@{}lll@{}}
\toprule
ID & Título & PH\tabularnewline
\midrule
\endhead
1 & Usuario - vincular la aplicación con servicios externos &
2\tabularnewline
8 & Usuario - registrarse & 3\tabularnewline
9 & Usuario - recibir notificaciones sobre servicios marcados &
2\tabularnewline
3 & Usuario - obtener ruta óptima para cada servicio & 4\tabularnewline
\bottomrule
\end{longtable}

\subsection{5. Tarjetas de las HU}\label{tarjetas-de-las-hu}

\end{document}
